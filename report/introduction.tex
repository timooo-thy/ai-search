\chapter{Introduction}
Software Engineering is undergoing a fundamental shift with the rise in Large Language Models (LLMs), Agentic Artificial Intelligence (AI), and generative user interfaces (GenUI). As the complexity of current software architectures (distributed systems) and codebases grow, the challenge of developers learning a new repository has become a significant bottleneck for engineering productivity. I propose CodeOrient, an autonomous AI-driven developer onboarding platform that leverages Retrieval-Augmented Generation (RAG) with dynamic graph visualisation. Deployed as an interactive onboarding assistant, CodeOrient combines the reasoning capabilities of LLMs with the structural insights of code graphs to transform how developers understand and navigate unfamiliar codebases.
\section{Motivation and Problem Statement}
The rapid advancement of AI-assisted coding tools, such as GitHub Copilot and Cursor, has prioritised code generation over code comprehension. As a new developer onboards, they often struggle with understanding the existing codebases due to their complex architectural nature. Three critical issues affecting effective onboarding are:

\begin{enumerate}
    \item \textbf{LLM Hallucinations:} LLMs are trained on past data and often generate plausible but factually incorrect information. This is particularly problematic as the generated response has no knowledge of the specific codebase being queried.
    \item \textbf{Limited Context Window:} Large codebases often exceed the input context window of LLMs, leading to incomplete or incorrect answers as the model cannot access all relevant information.
    \item \textbf{Overloading of Information:} Most codebases are accompanied with static documentation which fails to capture the dynamic relationships within the code. Without visual context, developers are often overwhelmed by the volume of code and struggle to identify relevant components in a large codebase.
\end{enumerate}

\section{Research Objectives and Goals}
The primary goal of this research is to design and implement CodeOrient, an AI-driven developer onboarding tool that addresses the challenges of code comprehension in complex codebases. The specific objectives are:

\begin{enumerate}
    \item \textbf{Develop a Retrieval-Augmented Generation (RAG) Framework:} Implement a hybrid search mechanism that combines vector-based semantic search with traditional keyword-based search to retrieve relevant code snippets and documentation.
    \item \textbf{Implement Code Visualisation with Generative UI:} Utilise React Flow to dynamically render interactive graphs based on user queries.
    \item \textbf{Minimise AI Hallucinations:} Ensure every response or claim made by the LLM is backed by code citations from the retrieved documents.
    \item \textbf{Evaluate the Effectiveness of Dynamic Visualisation:} Evaluate how dynamic graph visualisations affect the comprehension ability of developers compared to traditional text-based documentation.
\end{enumerate}

\section{Key Contributions}
CodeOrient introduces three key contributions that separates it from existing developer onboarding tools:

\begin{itemize}
    \item \textbf{Dynamic UIs:} Instead of traditional text-based responses generated by LLMs, Generative UI is utilised to create dynamic visualisations based on user queries. If a developer asks about "authentication flow," the UI creates a graph focused strictly on those related modules, rather than a cluttered, static diagram.
    \item \textbf{Citation-Grounded RAG Pipeline:} To reduce hallucination, CodeOrient integrates sources and inline citations mechanisms into the RAG framework, ensuring that all LLM responses can be verified against actual code segments.
    \item \textbf{Integration of Agentic LLMs:} CodeOrient utilises agentic LLMs equipped with tools. They can autonomously decide when to query the vector database, generate visualisations, or seek clarifications based on the conversation context.
\end{itemize}

The final outcome of this research is publicly accessible for demonstration at \url{https://codeorient.vercel.app}, and the source code is available at \url{https://github.com/timooo-thy/ai-search}.

\section{Report Structure}
This report is structured for the ideation to implementation journey of CodeOrient:

\begin{itemize}
    \item \textbf{Chapter 2:} Literature Review explores software complexity metrics, semantic code search, and the emerging technologies of Generative UI.
    \item \textbf{Chapter 3:} System Design and Architecture details the technical stack of CodeOrient which features the RAG pipeline, code graph generation, and the interactive chat interface.
    \item \textbf{Chapter 4:} Implementation Details discusses the development methodology, key algorithms, and integration challenges encountered during the build process.
    \item \textbf{Chapter 5 \& 6:} Evaluation, Results and Discussion will analyse the tool’s effectiveness to improve the onboarding experience through user studies and case studies on real-world codebases.
    \item \textbf{Chapter 7:} Conclusion summarises the contributions of this research and reflects on the transformative potential of AI-driven developer onboarding.
\end{itemize}